\chapter{Спряжений оператор}

\section{Алгебраїчно спряжний і спряжений оператори}

\begin{definition}
    Нехай~$X, Y$~--- лінійні простори, $T: X \to Y$~--- лінійний оператор. \vocab{Алгебраїчно спряженим оператором} до~$T$ називається оператор~$T': Y' \to X'$, що діє за правилом~$T' f = f \circ T$.
\end{definition}

\begin{definition}
    Нехай~$(X_1, Y_1)$, $(X_2, Y_2)$~--- дуальні пари. Будемо говорити, що у оператора~$T$ існує \vocab{спряжений оператор}~$\conjugate T: Y_2 \to Y_1$, якщо для будь-якого~$y \in Y_2$ існує такий елемент~$\conjugate T y \in Y_1$, що~$\langle T x, y \rangle = \langle x, \conjugate T, y \rangle$ для всіх~$x_1 \in X$.
\end{definition}

Трактуючи елементи просторів~$Y_1$, $Y_2$ як функціонали на~$X_1$ і~$X_2$ відповідно, бачимо, що~$\conjugate T y = y \circ T$. Очевидно, що спряжений оператор до~$T$ існує тоді і лише тоді, коли~$T'(Y_2) \subset Y_1$. У цьому випадку~$\conjugate T$~--- це звуження алгебраїчно спряженого оператора~$T'$ на~$Y_2$. Для дуальних пар~$(X_1, \conjugate X_1)$, $(X_2, \conjugate X_2)$, де~$X_1, X_2$~--- банахові простори, то нове означення спряженого оператора збігається з відомим означенням спряженого до оператору в банахових просторах.

\begin{theorem}
    Нехай~$X_1$ і~$X_2$~--- локально опуклі простори, $T: X_1 \to X_2$~--- лінійний неперервний оператор. Тоді у~$T$ існує спряжений~$\conjugate T: \conjugate X_2 \to \conjugate X_1$.
\end{theorem}

\begin{proof}
    Нехай~$f \in \conjugate X_2$. Тоді функціонал~$T' f = f \circ T$ є неперервним як композиція двох неперервних відображень. Отже, $T'(\conjugate X_2) \subset \conjugate X_1$.
\end{proof}

\begin{theorem}
    Нехай~$(X_1, Y_1)$, $(X_2, Y_2)$~--- дуальні пари, $T: X_1 \to X_2$~--- лінійний оператор, $\conjugate T: Y_2 \to Y_1$~--- спряжений оператор. Тоді для довільного~$A \subset Y_2$
    \begin{equation*}
        T\inv(A\polar) = (\conjugate T A)\polar.
    \end{equation*}
\end{theorem}

\begin{proof}
    \begin{multline*}
        x \in T\inv(A\polar) \iff T x \in A\polar \iff \forall y \in A \: |\langle T x, y\rangle| \le 1 \iff \\
        \iff \forall y \in A \: |\langle x, \conjugate T y\rangle| \le 1 \iff \forall z \in \conjugate T A \: |\langle x, z\rangle| \le 1 \iff x \in (\conjugate T A)\polar. \qedhere
    \end{multline*}
\end{proof}

\begin{theorem}
    Нехай~$(X_1, Y_1)$, $(X_2, Y_2)$~--- дуальні пари, $T: X_1 \to X_2$~--- лінійний оператор. Тоді наступні умови є еквівалентними:
    \begin{enumerate}
        \item У оператора~$T$ існує спряжений.
        \item $T$~--- слабко неперервний оператор, тобто він є неперервним як оператор, що діє з~$(X_1, \sigma(X_1, Y_1))$ і~$(X_2, \sigma(X_2, Y_2))$.
    \end{enumerate}
\end{theorem}

\begin{proof}
   ~$1 \implies 2$. Внаслідок лінійності достатньо перевірити неперервність оператору в нулі. За теоремою 13.2 базу околів нуля в топології~$\sigma(X_2, Y_2)$ утворюють поляри скінчених множин~$A \subset Y$. За формулою
    \begin{equation*}
        T\inv(A\polar) = (\conjugate T A)\polar
    \end{equation*}
    прообраз~$T\inv(A\polar)$ околу~$A\polar$~--- це знову поляра~$(\conjugate T A)\polar$ скінченої множини~$\conjugate T A \subset Y_1$. Отже, $T\inv(A\polar)$~--- це окіл нуля в топології~$\sigma(X_1, Y_1)$.
\end{proof}

\begin{corollary}
    Нехай~$X_1, X_2$~--- локально опуклі простори, $T: X_1 \to X_2$~--- лінійний неперервний оператор. Тоді~$T$~--- слабко неперервний оператор в топологіях~$\sigma(X_1, \conjugate X_1)$ і~$\sigma(X_2, \conjugate X_2)$.
\end{corollary}

\section{Література}

\begin{enumerate}[label={[\arabic*]}]
\item \textbf{Кадец~В.~М.}
Курс функционального анализа /
В.~М.~Кадец~---
Х.: ХНУ им.~В.~Н.~Каразина, 2006. (стр.~535--538).
\end{enumerate}
