\chapter{Теорема про ізоморфізм}

Обравши в~$n$-вимірному евклідовому просторі ортогональний нормований базис~$e_1, e_2, \dots, e_n$, можна кожний вектор~$x \in \RR^n$ записати у вигляді
\begin{equation*}
    x = \sum_{k = 1}^n c_k e_k,
\end{equation*}
де 
\begin{equation*}
    c_k = (x, e_k).
\end{equation*}

Постає питання, як узагальнити цей розклад на випадок нескінченновимірного евклідова простору. Введемо наступні поняття.

\section{Базиси у гільбертових просторах}

\begin{definition}
    Система ненульових векторів~$\{e_k\} \subset E$ називається \vocab{ортогональною}, якщо~$(e_k, e_l) = 0$ при~$k \ne l$.
\end{definition}

\begin{definition}
    Система~$\{e_k\} \subset E$, елементи якої задовольняють умову
    \begin{equation*}
        (e_k, e_l) = \begin{cases}
            0, & k \ne l, \\
            1, & k = l.
        \end{cases}
    \end{equation*}
    називається \vocab{ортонормованою}.
\end{definition}

Нагадаємо означення із теорії лінійних просторів.

\begin{definition}
    Найменший лінійний підпростір, що містить множину~$A$ у лінійному просторі~$X$, називається \vocab{лінійною оболонкою} множини~$A$, або лінійним підпростором, що породжений множиною~$A$. Цей підпростір позначається як~$\spanning A$.
\end{definition}

\begin{remark}
    Лінійна оболонка лінійної множини~$A$ є замкненою, але якщо множина~$A$ є довільною, це не обов'язково так. В той же час у нормованих просторах підпростори є замкненими за означенням, тому лінійна оболонка множини в нормованому просторі є замкненою.
\end{remark}

\begin{definition}
    Система~$\{e_k\} \subset E$ називається \vocab{повною}, якщо її лінійна оболонка є скрізь щільною в~$E$, тобто~$\closure{\spanning \{e_k\}} = E$.
\end{definition}

\begin{definition}
    Повна ортонормована система~$\{e_k\} \subset E$ називається \vocab{ортонормованим базисом}.
\end{definition}

\begin{example}
    В просторі~$\ell_2$ ортонормований базис утворюють послідовності
    \begin{equation*}
        e_i = (\underset{i - 1}{\underbrace{0, \dots, 0}}, 1, 0, \dots).
    \end{equation*}
    
    Скалярний добуток: $(x, y) = \sum_{n = 1}^\infty x_n y_n$.
\end{example}

\begin{example}  % wrong
    В просторі~$C^2(a, b)$ ортонормований базис утворюють вектори
    \begin{equation*}
        \frac{1}{2}, \cos \frac{2 \pi t}{b - a}, \sin \frac{2 \pi t}{b - a}, \dots
        \cos \frac{2 \pi n t}{b - a}, \sin \frac{2 n \pi t}{b - a}, \dots
    \end{equation*}
    
    Скалярний добуток: $(f, g) = \int_a^b f(t) g (t) dt$.
\end{example}

\begin{lemma}
    В сепарабельному евклідовому просторі будь-яка ортогональна система є не більш ніж зліченною.
\end{lemma}

\begin{proof}
    Не обмежуючи загальності, розглянемо ортонормовану систему~$\{\phi_k\} \subset E$. Тоді
    \begin{multline*}
        \|\phi_k - \phi_l\| =
        \sqrt{(\phi_k - \phi_l, \phi_k - \phi_l)} = \\
        \sqrt{(\phi_k, \phi_k) - 2 (\phi_k, \phi_l) + (\phi_l, \phi_l)} = \\
        \sqrt{(\phi_k, \phi_k) + (\phi_l, \phi_l)} =
        \sqrt{1 + 1} = \sqrt{2}.
    \end{multline*}
    
    Розглянемо сукупність куль~$S\left( \phi_k, \frac{1}{2} \right)$. Ці кулі не перетинаються. Якщо зліченна множина~$\{\psi_k\}$ є скрізь щільною в~$E$, то в кожну кулю потрапить принаймні один елемент~$\psi_k$. Отже, потужність множини таких куль не може перевищувати потужність зліченої множини. 
\end{proof}

\section{Елементи аналізу Фур'є}

\begin{definition}
    Ортонормована система~$\{\phi_k\} \subset E$ називається \vocab{замкненою}, якщо для довільного~$f \in E$ виконується \vocab{рівність Парсеваля}
    \begin{equation}
        \sum_{k = 1}^\infty c_k^2 = \|f\|^2.
    \end{equation}
\end{definition}

\begin{definition}
    Нехай~$\{\phi_k\} \subset E$~--- ортонормована система в евклідовому просторі, а~$f$~--- довільний елемент із~$E$. Поставимо у відповідність елементу~$f \in E$ послідовність чисел
    \begin{equation*}
        c_k = (f, \phi_k), \quad k = 1, 2, \dots
    \end{equation*}
    
    Числа~$c_k$ називаються \vocab{координатами}, або \vocab{коефіцієнтами Фур'є} елемента~$f$ по системі~$\{\phi_k\} \subset E$, а ряд
    \begin{equation*}
        \sum_{k = 1}^\infty c_k \phi_k
    \end{equation*}
    називається \vocab{рядом Фур'є} елемента~$f$ по системі~$\{\phi_k\} \subset E$.
\end{definition}

\begin{theorem}
    Ряд Фур'є збігається тоді і лише тоді, коли система~$\{\phi_k\} \subset E$ є замкненою.
\end{theorem}

\begin{proof}
    Розглянемо суму
    \begin{equation*}
        S_n = \sum_{k = 1}^n \alpha_k \phi_k
    \end{equation*}
    і для заданого числа~$n$ відшукаємо коефіцієнти~$\alpha_k$, що мінімізують~$\|f - S_n\|^2$.
    \begin{multline*}
        \|f - S_n\|^2 =
        \left( f - \sum_{k = 1}^n \alpha_k \phi_k, f - \sum_{k = 1}^n \alpha_k \phi_k \right) = \\
        (f, f) - 2 \left( f, \sum_{k = 1}^n \alpha_k \phi_k \right) + \left( \sum_{k = 1}^n \alpha_k \phi_k, \sum_{k = 1}^n \alpha_k \phi_k \right) = \\
        \|f\|^2 - 2 \sum_{k = 1}^n \alpha_k c_k + \sum_{k = 1}^n \alpha_k^2 =
        \|f\|^2 - \sum_{k = 1}^n c_k^2 + \sum_{k = 1}^n (\alpha_k - c_k)^2.
    \end{multline*}

    Мінімум цього виразу досягається тоді, коли останній член дорівнює нулю, тобто, коли
    \begin{equation*}
        \alpha_k = c_k, \quad k = 1, 2, \dots, n.
    \end{equation*}
    
    В цьому випадку
    \begin{equation}
        \label{eq:17.3}
        \|f - S_n\|^2 = \|f\|^2 - \sum_{k = 1}^n c_k^2.
    \end{equation}
    
    Оскільки~$\|f - S_n\|^2 \ge 0$, то
    \begin{equation*}
        \sum_{k = 1}^n c_k^2 \le \|f\|^2.
    \end{equation*}
    
    Переходячи до границі при~$n \to \infty$, отримуємо нерівністю Бесселя:
    \begin{equation*}
        \sum_{k = 1}^\infty c_k^2 \le \|f\|^2.
    \end{equation*}

    Із тотожності \eqref{eq:17.3} випливає, що рід Фур'є збігається тоді і лише тоді, коли виконується рівність Парсеваля, тобто система є замкненою. 
\end{proof}

\begin{theorem}[Рісса---Фішера]
    Нехай~$\{\phi_k\} \subset E$~--- довільна ортонормована система в гільбертовому просторі~$E$, а числа~$c_1, c_2, \dots, c_n, \dots$ є такими, що ряд~$\sum_{k = 1}^n c_k^2$ є збіжним.

    Тоді існує такий елемент~$f \in E$, що~$c_k = (f, \phi_k)$ і~$\sum_{k = 1}^n c_k^2 = (f, f) = \|f\|^2$.
\end{theorem}

\begin{proof}
    Розглянемо суму
    \begin{equation*}
        f_n = \sum_{k = 1}^n c_k \phi_k.
    \end{equation*}
    
    Тоді,
    \begin{equation*}
        \|f_{n + p} - f_n\|^2 =
        \|c_{n + 1} \phi_{n + 1} + \dots + c_{n + p} \phi_{n + p}\|^2 =
        \sum_{k = n + 1}^{n + p} c_k^2.
    \end{equation*}

    Оскільки ряд~$\sum_{k = 1}^n c_k^2$ є збіжним, а простір~$E$~--- повним, послідовність~$\{f_n\}_{n = 1}^\infty$ збігається до деякого елемента~$f \in E$. Оцінимо наступний скалярний добуток.
    \begin{equation*}
        (f, \phi_i) = (f_n, \phi_i) + (f - f_n, \phi_i).
    \end{equation*}
    
    При~$n \ge i$ перший доданок дорівнює~$c_i$, а другий доданок при~$n \to \infty$ прямує до нуля, оскільки
    \begin{equation*}
        |(f - f_n, \phi_i)| \le \|f - f_n\| \cdot \|\phi_i\|.
    \end{equation*}
    
    Ліва частина рівності від~$n$ не залежить. Переходячи до границі при~$n \to \infty$, доходимо висновку, що
    \begin{equation*}
        (f, \phi_i) = c_i.
    \end{equation*}
    
    Оскільки за означенням елемента~$f$
    \begin{equation*}
        \lim_{n \to \infty} \|f - f_n\| = 0,
    \end{equation*}
    то
    \begin{equation*}
        \left( f - \sum_{k =1}^n c_k \phi_k, f - \sum_{k =1}^n c_k \phi_k \right) =
        (f, f) - \sum_{k = 1}^\infty c_k^2 \to 0, \quad n \to \infty.
    \end{equation*}
    
    Отже,
    \begin{equation*}
        \sum_{k = 1}^n c_k^2 = (f, f). \qedhere
    \end{equation*}
\end{proof}

\section{Сепарабельний простір}

\begin{theorem}
    В сепарабельному евклідовому просторі~$E$ будь-яка повна ортонормована система є замкненою, і навпаки.
\end{theorem}

\begin{proof}
    \textbf{Необхідність.} Нехай система~$\{\phi_k\} \subset E$ є замкненою. Тоді за \error теоремою 17.1 для довільного елемента~$f \in E$ послідовність часткових сум його ряду Фур'є збігається до~$f$. Це означає, що~$\closure{\spanning \{\phi_k\}} = E$, тобто система~$\{\phi_k\}$ є повною.

    \textbf{Достатність.} Нехай система~$\{\phi_k\}$ є повною, тобто довільний елемент~$f \in E$ можна скільки завгодно точно апроксимувати лінійною комбінацією~$\sum_{k = 1}^n \alpha_k \phi_k$ елементів системи~$\{\phi_k\}$:
    \begin{equation*}
        \forall \epsilon > 0: \exists \sum_{k = 1}^n \alpha_k \phi_k:
        \left\| f - \sum_{k = 1}^n \alpha_k \phi_k \right\| < \epsilon.
    \end{equation*}
    
    За \error теоремою 17.1 елементом найкращого наближення серед усіх сум вигляду~$\sum_{k = 1}^n \alpha_k \phi_k$ є ряд Фур'є. Отже, цей ряд збігається, а, значить, виконується рівність Парсеваля, тобто система~$\{\phi_k\}$ є замкненою.
\end{proof}

\begin{theorem}[про ізоморфізм]
    Довільні два сепарабельних гільбертових простора є ізоморфними один до одного.
\end{theorem}

\begin{proof}
    Покажемо, що кожний гільбертів простір~$H$ є ізоморфним простору~$\ell_2$. Це доведе теорему про ізоморфізм.
    
    Виберемо в~$H$ довільну повну ортонормовану систему~$\{\phi_k\} \subset H$ і поставимо у відповідність елементу~$f \in H$ сукупність його коефіцієнтів Фур'є за цією системою~$c_1, c_2, \dots, c_n, \dots$. Оскільки~$\sum_{k = 1}^\infty c_k^2 < \infty$, то послідовність~$\{c_1, c_2, \dots, c_n, \dots\}$ належить~$\ell_2$.
    
    І навпаки, за теоремою Рісса---Фішера довільному елементу~$\{c_1, c_2, \dots, c_n, \dots\} \in \ell_2$ відповідає деякий елемент~$f \in H$, у якого числа~$c_1, c_2, \dots, c_n, \dots$ є коефіцієнтами Фур'є за системою~$\{\phi_k\} \subset H$. Ця відповідність є взаємно-однозначною.
    
    Крім того, якщо
    \begin{align*}
        f &\leftrightarrow \{c_1, c_2, \dots, c_n, \dots\}, \\
        g &\leftrightarrow \{d_1, d_2, \dots, d_n, \dots\},
    \end{align*}
    то
    \begin{align*}
        f + g &\leftrightarrow \{c_ 1 + d_1, c_2 + d_2, \dots, c_n + d_n, \dots\}, \\
        \alpha f &\leftrightarrow \{\alpha c_1, \alpha c_2, \dots, \alpha c_n, \dots\}.
    \end{align*}
    
    Нварешті, із рівності Парсеваля випливає, що
    \begin{equation*}
        (f, f) = \sum_{k = 1}^\infty c_k^2, \quad
        (g, g) = \sum_{k = 1}^\infty d_k^2,
    \end{equation*}
    а тому
    \begin{equation*}
        2 (f, g) = (f + g, f + g) - (f, f) - (g, g) =
        \sum_{k = 1}^\infty (c_k + d_k)^2 -
        \sum_{k = 1}^\infty c_k^2 -
        \sum_{k = 1}^\infty d_k^2 =
        2 \sum_{k = 1}^\infty c_k d_k.
    \end{equation*}
    
    Отже,
    \begin{equation*}
        (f, g) = \sum_{k = 1}^\infty c_k d_k.
    \end{equation*}
    
    Таким чином, установлена відповідність між елементами просторів~$H$ і~$\ell_2$ є ізоморфізмом. 
\end{proof}

\section{Література}

\begin{enumerate}[label={[\arabic*]}]
\item \textbf{Колмогоров~А.~Н.}
Элементы теории функций и функционального анализа. 5-е изд. /
А.~Н.~Колмогоров, С.~В.~Фомин ---
М.: Наука, 1981 (стр.~149--157).
\end{enumerate}
