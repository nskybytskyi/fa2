\section*{Про цю книгу}

Цей матеріал позиціонується як спроба покращити
структурованість, зрозумілість,
типографічну та естетичну якість
конспекту лекцій
Дмитра Анатолійовича Клюшина,
а саме його курсів ``Функціональний аналіз''
і ``Додаткові розділи функціонального аналізу'',
що викладаються у четвертому та дев'ятому семестрах відповідно,
на спеціальності ``Прикладна математика''
на факультеті комп'ютерних наук та кібернетики (ФКНК)
київського національного університут імені Тараса Шевченка (КНУ).

Книга структурована наступним чином:
кожна глава представляє собою одну лекцію
і містить кілька розділів.
Зокрема, наприкінці кожної глави
є рекомендована література до відповідної лекції,
а також кілька типових або цікавих задач
для закріплення здобутих знань на практиці.
Самі глави зібрані у частини
які приблизно розділяють матеріал на змістовні модулі.
