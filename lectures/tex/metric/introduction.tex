\chapter{Метричні простори}

Численні поняття і теореми математичного аналізу
використовують поняття відстані між точками простору.
Зокрема, це стосується границі і неперервності. В багатьох
випадках самі теореми та їх доведення залежать не від
способу завдання метрики, а лише від їхніх властивостей:
невід’ємності, симетрії і нерівності трикутника.

\section{Основні означення}

\begin{definition}
Нехай X — довільна множина. Відображення
$\rho: X \times X \to \RR^+$ називається \vocab{метрикою}, якщо~$\forall x, y, z \in X$
воно має такі властивості (аксіоми метрики):
\begin{enumerate}
    \item $\rho(x, y) = 0 \iff x = y$ (аксіома тотожності);
    \item $\rho(x, y) = \rho(y, x)$ (аксіома симетрії);
    \item $\rho(x, y) \le \rho(x, z) + \rho(z, y)$ (нерівність трикутника).
\end{enumerate}
\end{definition}

\begin{definition}
\vocab{Метричним простором} називається пара
$(X, \rho)$, де~$X$~--- множи\-на-носій, а~$\rho$~--- метрика.
\end{definition}

\begin{example}
$\left( \RR^n, \sqrt{\sum_{i = 1}^n (x_i - y_i)^2} \right)$.
\end{example}

\begin{example}
$\left( C[a, b], \max_{t \in [a, b]} |x(t) - y(t)| \right)$.
\end{example}

\begin{definition}
\vocab{Відкритою кулею} радіуса~$\epsilon > 0$ з центром в
точці~$x_0 \in X$ називається множина
\begin{equation*}
    S(x_0, \epsilon) = \{x \in X: \rho(x, x_0) < \epsilon \}.
\end{equation*}
\end{definition}

\begin{definition}
\vocab{Замкненою кулею} радіуса~$\epsilon > 0$ з центром в
точці~$x_0 \in X$ називається множина
\begin{equation*}
    \closure S(x_0, \epsilon) = \{x \in X: \rho(x, x_0) \le \epsilon \}.
\end{equation*}
\end{definition}

\begin{example}
В просторі~$(\RR, |x - y|)$ відкритою кулею~$S(x_0, r)$
є інтервал~$(x_0 - r, x_0 + r)$, а замкненою кулею ---
сегмент~$[x_0 - r, x_0 + r]$.
\end{example}

\begin{example}
В просторі~$(\RR^2, \sqrt{(x_1 - y_1)^2 + (x_2 - y_2)^2})$ відкритою
кулею~$S(x_0, r)$ є круг без границі радіуса~$r$ з центром в точці~$x_0$.
\end{example}

\begin{example}
В просторі~$(\RR^2, |x_1 - y_1| + |x_2 - y_2|)$ одинична
куля є ромбом з вершинами~$(0, 1)$, $(1, 0)$, $(0, -1)$ і~$(-1, 0)$.
\end{example}

\begin{example}
В просторі~$\left( C[a, b], \max_{t \in [a, b]} |x(t) - y(t)| \right)$
околом є смуга, що складається із функцій, які
задовольняють умові~$\forall t \in [a, b]$: $|x(t) - y (t)| < r$.
\end{example}

\begin{definition}
Множина~$G \subset  X$ називається \vocab{відкритою} в
метричному просторі~$(X, \rho)$, якщо~$\forall x \in G$~$\exists S(x, r) \subset G$.
\end{definition}

\begin{definition}
Множина~$G \subset X$ називається \vocab{замкненою}, якщо
її доповнення є відкритою множиною.
\end{definition}

\begin{definition}
Множина метричного простору є \vocab{обмеженою
за відстанню}, або просто \vocab{обмеженою}, якщо воно
міститься в деякій кулі: $\exists S(x, r): M \subset S(x, r)$.
\end{definition}

\section{Збіжність і замкненість}

\begin{definition}
Точка~$x$ метричного простору~$(X, \rho)$
називається \vocab{границею послідовності} точок~$x_n \in X$, якщо
$\rho(x_n, x) \to 0$  при~$n \to \infty$. Така збіжність називається
\vocab{збіжністю за відстанню} (або за метрикою).

Цей факт записується так: $x = \lim_{n \to \infty} x_n$.
\end{definition}

\begin{lemma}
Для довільних точок~$x, x', y, y'$ метричного
простору~$(X, \rho)$ виконується нерівність
\begin{equation*}
    |\rho(x', y') - \rho(x, y)| \le \rho(x, x') + \rho(y, y').
\end{equation*}
\end{lemma}

\begin{proof}
Із нерівності трикутника випливає:
\begin{equation*}
    \rho(x', y') \le
    \rho(x', x) + \rho(x, y') \le
    \rho(x, x') + \rho(x, y) + \rho(y, y').
\end{equation*}
Отже,
\begin{equation*}
    \rho(x', y') - \rho(x, y) \le
    \rho(x, x') + \rho(y, y').
\end{equation*}
Аналогічно,
\begin{equation*}
    \rho(x, y) \le
    \rho(x, x') + \rho(x', y) \le
    \rho(x, x') + \rho(x', y') + \rho( y', y).
\end{equation*}
Звідси випливає, що
\begin{equation*}
    \rho(x, y) - \rho(x', y') \le
    \rho(x, x') + \rho(y, y').
\end{equation*}
Таким чином,
\begin{equation*}
    |\rho(x', y') - \rho(x, y)| \le
    \rho(x, x') + \rho(y, y'). \qedhere
\end{equation*}
\end{proof}

\begin{lemma}
Метрика~$\rho(x, y)$ є неперервною функцію своїх
аргументів, тобто якщо~$x_n \to x$, $y_n \to y$, то
$\rho(x_n, y_n) \to \rho(x, y)$.
\end{lemma}

\begin{proof}
Із леми 6.1 випливає, що при~$x_n \to x$, $y_n \to y$
\begin{equation*}
    |\rho(x_n, y_n) - \rho(x, y)| \le
    \rho(x_n, x) + \rho(y_n, y) \to
    0. \qedhere
\end{equation*}
\end{proof}

\begin{theorem}
Відкрита куля~$S(a, r)$ в метричному
просторі~$(X, \rho)$ є відкритою множиною в топології
метричного простору, що породжена його метрикою.
\end{theorem}

\begin{proof}
Розглянемо довільну точку~$x \in S(a, r)$.
\begin{equation*}
    x \in S(a, r) \implies
    \rho(x, a) < r.
\end{equation*}
Покладемо~$\epsilon = r - \rho(x, a)$. Розглянемо довільну точку
$y \in S(x, \epsilon)$.
\begin{equation*}
    y \in S(x, e) \implies
    \rho(y, x) < \epsilon.
\end{equation*}
\begin{equation*}
    \rho(y, a) \le
    \rho(y, x) + \rho(x, a) < r \implies
    y \in S(a, r) \implies S(x, \epsilon) \subset S(a, r).
\end{equation*}
Таким чином, точка~$x$ є внутрішньою точкою множини
$S(a, r)$, тобто~$S(a, r)$~--- відкрита множина. 
\end{proof}

\begin{theorem}
Точка~$x$ належить замиканню~$\closure A$ множини
$A \subset X$ в топології, що індукована на~$X$ метрикою~$\rho$, тоді і
лише тоді, якщо існує послідовність точок множини~$A$, що
збігається до точки~$x$.
\end{theorem}

\begin{proof}
Необхідність.
\begin{equation*}
    x \in \closure A \implies
    \forall n \in \NN \exists x_n \in S \cap S(x, \tfrac{1}{n}) \implies
    \rho(x, x_n) < \tfrac{1}{n} \implies
    x = \lim_{n \to \infty} x_n.
\end{equation*}

Достатність.
\begin{multline*}
    x \not\in \closure A \implies
    \exists r > 0: A \cap S(x, r) = \emptyset \implies \\
    \forall x' \in A: \rho(x, x') \ge r \implies
    \nexists \{x_n\}_{n \in \NN} \subset A: \lim_{n \to \infty} x_n = x.
\end{multline*}

Що і треба було довести.
\end{proof}

\begin{corollary}
Теорема 6.2 стверджує, що кожна точка
дотику множини в метричному просторі є границею деякої
послідовності елементів цієї множини. Отже, топологію
метричного простору можна описати не лише за
допомогою куль, а й за допомогою збіжних послідовностей.
\end{corollary}

\begin{corollary}
Множина є замкненою, якщо всі
послідовності її точок збігаються лише до точок цієї ж
множини.
\end{corollary}

\begin{theorem}
Замкнена куля~$\closure S(a, r)$ є замкненою
множиною в топології метричного простору, що
породжена його метрикою.
\end{theorem}

\begin{proof}
Нехай~$x_n \in \closure S(a, r)$.
\begin{multline*}
    x_n \in \closure S(a, r) \implies
    \rho(x_n, a) \le r \implies \\
    \lim_{n \to \infty} \rho(x_n, a) = \rho\left(\lim_{n \to \infty} x_n, a\right) = \rho(x, a) \le r \implies
    x \in \closure S(a, r).
\end{multline*}

Отже, всі граничні точки множини~$\closure S(a, r)$, які є точками
її дотику, належать кулі~$\closure S(a, r)$. 
\end{proof}

\section{Збіжність і фундаментальність}

\begin{definition}
Послідовність~$\{x_n\}_{n \in \NN}$ точок метричного
простору~$(X, \rho)$ називається \vocab{фундаментальною}, якщо
$\rho(x_n, x_m) \to 0$ при~$n \to \infty$, $m \to \infty$.
\end{definition}

\begin{lemma}
Будь-яка збіжна послідовність метричного
простору є фундаментальною.
\end{lemma}

\begin{proof}
Нехай~$x_n \to x$ при~$n \to \infty$. Тоді
\begin{equation*}
    \rho(x_n, x_m) \le \rho(x_n, x) + \rho(x, x_m) \to 0 \text{ при } n, m \to \infty.
\end{equation*}
Отже, послідовність є фундаментальною. 
\end{proof}

\begin{lemma}
Будь-яка фундаментальна послідовність
точок метричного простору є обмеженою.
\end{lemma}

\begin{proof}
Задамо~$\epsilon > 0$ і підберемо натуральне число~$N$
так, щоб~$\rho(x_n, x_m) < \epsilon$ при~$n, m \ge N$.
Зокрема, $\rho(x_n, x_N) < \epsilon$ при~$n \ge N$.
Введемо позначення
\begin{equation*}
    r = \max \{ \epsilon, \rho(x_1, x_N), \rho(x_2, x_N), \dots, \rho(x_{N - 1}, x_N) \}.
\end{equation*}

Тепер при всіх~$n = 1, 2, \dots$
\begin{equation*}
    \rho(x_n, x_N) \le r.
\end{equation*}

Інакше кажучи,
\begin{equation*}
    \{x_n\}_{n \in \NN} \subset \closure S(x_N, r).
\end{equation*}
Замінюючи число~$r$ на будь-яке число~$r' > r$, можна
заключити послідовність в довільну відкриту кулю:
\begin{equation*}
    \{x_n\}_{n \in \NN} \subset S(x_N, r'). \qedhere
\end{equation*}
\end{proof}

\section{Література}

\begin{enumerate}[label={[\arabic*]}]
\item \textbf{Александрян~Р.~А.}
Общая топология /
Р.~А.~Александрян, Э.~А.~Мирзаханян ---
М.: Высшая школа, 1979 (стр.~47--50).
\item \textbf{Садовничий~В.~А.}
Теория операторов /
В.~А.~Садовничий ---
М.: Изд-во Моск. ун-та, 1986 (стр.~60--69).
\end{enumerate}
