\newcommand{\InducedNeighbourhood}[1]{U_{n,\{f_k\}_{k=1}^n,\{V_k\}_{k=1}^n}(#1)}

\chapter{Топологія, що породжена сім'єю відображень}

\section{Топологія, у якій задані функції неперервні}

Нехай на множині~$X$ задано сім'я відображень~$F$, де відображення~$f \in F$ діють у топологічні простори~$f(X)$, які, взагалі кажучи, можуть бути різними. Для будь-якої точки~$x \in X$, будь-якого скінченного сім'ї відображень~$\{f_k\}_{k = 1}^n \subset F$ і відкритих околів~$V_k$ точок~$f_k(x)$ в просторі~$f_k(X)$ визначимо множини
\begin{equation*}
    \InducedNeighbourhood{x} = \Bigcap_{k = 1}^n f_k\inv(V_k).
\end{equation*}

Як відомо, якщо для кожної точки~$x \in X$ задане непорожнє сім'я підмножин~$U_x$, що має такі властивості:
\begin{enumerate}
    \item якщо~$U \in U_x$, то~$x \in U$;
    \item якщо~$U_1, U_2 \in U_x$, то існує таке~$U_3 \in U_x$, що~$U_3 \subset U_1 \cap U_2$;
    \item якщо~$U \in U_x$ і~$y \in U$, то існує така множина~$V \in U_y$, що~$V \subset U$,
\end{enumerate}
то існує топологія~$\tau$ на~$X$, для якої сім'ї~$U_x$ будуть базами околів відповідних точок. 

Таким чином, на~$X$ існує топологія, для якої множини~$\InducedNeighbourhood{x}$ утворюють базу околів точки~$x$ при всіх точках~$x \in X$. Позначимо цю топологію як~$\sigma(X, F)$. Зокрема, околами точок~$x \in X$ в топології~$\sigma(X, F)$ будуть всі множини~$f\inv(V)$, де~$f \in F$, а~$V$~--- окіл точки~$f(x)$ в топологічному просторі~$f(X)$. Отже, усі відображення сім'ї~$F$ є неперервними в топології~$\sigma(X, F)$.

\begin{theorem}
   ~$\sigma(X, F)$~--- найслабкіша топологія серед усіх топологій на~$X$, в яких усі відображення сім'ї~$F$ є неперервними.
\end{theorem}

\begin{proof}
    Нехай~$\tau$~--- довільна топологія, в якій усі відображення сім'ї~$F$ є неперервними. Доведемо, що будь-яка множина~$\InducedNeighbourhood{x}$ є околом точки~$x$ в топології~$\tau$. Звідси випливатиме, що~$\tau \succ \sigma(X, F)$. За умовою, усі відображення~$f_k: X \to F_k(X)$ є неперервними в топології~$\tau$. Отже, $f_k\inv(V_k)$~--- це відкриті околи точки~$x$ в топології~$\tau$. Відкритим околом буде і скінченний перетин~$\InducedNeighbourhood{x}$ таких множин.
\end{proof}

\begin{definition}
    Топологія~$\sigma(X, F)$ називається топологією, \vocab{породженою сім'єю відображень~$F$}, або \emph{слабкішою топологією, в якій усі відображення сім'ї~$F$ є неперервними}.
\end{definition}

\section{Породжена топологія і віддільність}

\begin{definition}
    Кажуть, що сім'я відображень~$F$ \vocab{розділяє} точки множини~$X$, якщо~$\forall x_1, x_2 \in X$, $x_1 \ne x_2$, $\exists f \in F$: $f(x_1) \ne f(x_2)$.
\end{definition}

\begin{theorem}
    \label{th:induced-topology-hausdorff-separability-criterion}
    Нехай усі простори~$f(X)$, $f \in F$ є хаусдорфовими. Для того щоб топологія~$\sigma(X, F)$ була віддільною за Хаусдорфом необхідно і достатньо, щоб сім'я відображень~$F$ розділяла точки множини~$X$.
\end{theorem}

\begin{proof}
    \textbf{Достатність.} Припустимо, що сім'я відображень~$F$ розділяє точки множини~$X$. Тоді
    \begin{equation*}
        \forall x_1, x_2 \in X, x_1 \ne x_2 \quad \exists f \in F: \quad f(x_1) \ne f(x_2).
    \end{equation*}

    Оскільки~$f(X)$~--- хаусдорфів простір, існують околи~$V_1, V_2$ точок~$f(x_1)$ і~$f(x_2)$ відповідно. Множини~$f\inv(V_1)$ і~$f\inv(V_2)$ є шуканими околами в топології~$\sigma(X, F)$, що розділяють точки~$x_1$ і~$x_2$.

    \textbf{Необхідність.} Нехай сім'я відображень~$F$ не розділяє точок множини~$X$. Тоді
    \begin{equation*}
        \exists x_1, x_2 \in X, x_1 \ne x_2 \quad \forall f \in F: \quad f(x_1) = f(x_2).
    \end{equation*}

    Візьмемо довільний окіл~$\InducedNeighbourhood{x_1}$ точки~$x_1$ в топології~$\sigma(X, F)$. Оскільки~$f_k(x_1) = f_k(x_2)$ для всіх~$k = 1, 2, \dots, n$, то й точка~$x_2$ лежить у тому ж околі~$\InducedNeighbourhood{x_1}$. Отже, в топології~$\sigma(X, F)$, не виконується навіть аксіома про віддільність, а не лише властивість Хаусдорфа.
\end{proof}

\section{Породжена топологія і фільтри}

\begin{theorem}
    Для того щоб фільтр~$\frak F$ на~$X$ збігався в топології~$\sigma(X, F)$ до елемента~$x$, необхідно і достатньо, щоб умова~$\lim_{\frak F} f = f(x)$ виконувалася для всіх~$f \in F$.
\end{theorem}
\begin{proof}
    \textbf{Необхідність.} З огляду на неперервність усіх~$f \in F$ в~$\sigma(X, F)$, необхідність випливає з теореми 3.3.

    \textbf{Достатність.} Нехай~$\lim_{\frak F} f = f(x)$ для всіх~$f \in F$. Доведемо, що будь-який окіл~$\InducedNeighbourhood{x}$ є елементом фільтра~$\frak F$. За умовою, $\lim_{\frak F} f_k = f_k(x)$, отже~$f_k\inv(V_k) \in \frak F$ для усіх~$k = 1, 2, \dots, n$. Оскільки фільтр є замкненим відносно скінченого перетину елементів
    \begin{equation*}
        \InducedNeighbourhood{x} = \Bigcap_{k = 1}^n f_k\inv(V_k) \in \frak F. \qedhere
    \end{equation*}
\end{proof}

\section{Література}

\begin{enumerate}[label={[\arabic*]}]
\item \textbf{Кадец~В.~М.}
Курс функционального анализа /
В.~М.~Кадец~---
Х.: ХНУ им.~В.~Н.~Каразина, 2006. (стр.~492--495).
\end{enumerate}
