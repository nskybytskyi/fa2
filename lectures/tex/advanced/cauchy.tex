\chapter{Повнота, передкомпактність, компактність}

\section{Фільтр Коші}

\begin{definition}
    Фільтр~$\frak F$ у топологічному векторному просторі~$X$ називається \vocab{фільтром Коші}, якщо для будьякого околу нуля~$U$ існує такий елемент~$A \in F$, що~$A - A \subset U$. Такий елемент~$A$ називається \vocab{малим порядку~$U$}.
\end{definition}

\begin{theorem}
    Якщо фільтр~$\frak F$ має границю, то~$\frak F$~--- фільтр Коші.
\end{theorem}

\begin{proof}
    Нехай~$\lim \frak F = x$ і~$U \in \frak R_0$. Виберемо~$V \in \frak R_0$ з~$V - V \subset U$. За \error теоремою 1.1 (п.~1) існує такий елемент~$A \in \frak F$, що~$A \subset x + V$ Отже,
    \begin{equation*}
        A - A \subset(x + V) -(x + V) \subset V - V \subset U. \qedhere
    \end{equation*}
\end{proof}

\begin{theorem}
    Нехай~$\frak F$~--- фільтр Коші на ТВП~$X$ і~$x$~--- гранична точка~$\frak F$. Тоді~$\lim \frak F = x$.
\end{theorem}

\begin{proof}
    Нехай~$x + U$~--- довільний окіл точки~$x$, де~$U \in \frak R_0$. Виберемо окіл~$V \in \frak R_0$ з~$V + V \subset U$ і множину~$A \in F$, малу порядку~$V$: $A - A \subset V$. За означенням граничної точки, множини~$A$ і~$x + V$ перетинаються, тобто існує~$y \in A \cap(x + V)$. Тоді
    \begin{equation*}
        x + U \supset x + V + V \supset y + V \supset y + A - A \supset y + A - y = A.
    \end{equation*}

    Таким чином, окіл~$x + U$ містить елемент фільтра~$\frak F$, отже, $x + U \in F$.
\end{proof}

\section{Повнота і фільтри}

\begin{definition}
    Множина~$A$ у ТВП~$X$ називається \vocab{повною}, якщо будь-який фільтр Коші на~$X$, що містить~$A$ як елемент, має границю, що належить~$A$.
\end{definition}

\begin{remark}
    Зокрема, топологічний векторний простір~$X$ називається \vocab{повним}, якщо будь-який фільтр Коші в~$X$ має границю.
\end{remark}

\begin{theorem}
    Нехай~$X$~--- підпростір топологічного векторного простору~$E$ і~$A \subset X$~--- повна в~$X$ підмножина. Тоді~$A$ є повною як підмножина простору~$E$.
\end{theorem}

\begin{proof}
    Нехай~$\frak F$~--- фільтр Коші на~$E$, що містить~$A$ як елемент. Тоді, зокрема~$X \in \frak F$, то слід~$\frak F_X$ фільтра~$\frak F$ на~$X$ є фільтром. Легко бачити, що~$\frak F_X$~--- це фільтр Коші на~$X$, що містить~$A$ як елемент. Отже, через повноту~$A$ у~$X$ фільтр~$\frak F_X$ має в~$X$ границю~$a \in A$. Ця ж точка~$a$ буде границею фільтра~$\frak F$ в~$E$.
\end{proof}

\begin{theorem}
    Повна підмножина~$A$ хаусдорфового ТВП~$X$ є замкнутою.
\end{theorem}

\begin{remark}
    Зокрема, якщо підпростір хаусдорфового ТВП є повним в індукованій топології, то цей підпростір є замкнутим.
\end{remark}

\section{Передкомпактність і компактність}

\begin{proof}
    Нехай точка~$x \in X$ належить замиканню множини~$A$. Нам потрібно довести, що~$x \in A$. Розглянемо сімейство~$\frak D$ усіх перетинів вигляду~$(x + U) \cap A$, де~$U \in \frak R_0$. Усі такі перетини не порожні, і~$\frak D$ задовольняє усі аксіоми бази фільтра. Фільтр~$\frak F$, породжений базою~$\frak D$, мажорує фільтр~$\frak R_x$ усіх околів точки~$x$, отже, $x = \lim \frak F$. Зокрема, $\frak F$~--- це фільтр Коші. За побудовою, наша повна множина~$A$ є елементом фільтра~$\frak F$; отже, відповідно до означення, фільтр~$\frak F$ має границю в~$A$. Через єдиність границі~$x \in A$, що і було потрібно довести.
\end{proof}

\begin{definition}
    Множина~$A$ у ТВП~$X$ називається \vocab{передкомпактом}, якщо для будь-якого околу нуля~$U$ існує така скінченна множина~$B \subset X$, що~$A \subset U + B$. Така множина~$B$ називається, за аналогією з~$\epsilon$-сіттю, \vocab{$U$-сіттю} множини~$A$.
\end{definition}

\begin{theorem}
    Щоб підмножина~$A$ хаусдорфового ТВП~$X$ була компактом, необхідно і достатньо, щоб~$A$ була одночасно передкомпактом і повною множиною в~$X$.
\end{theorem}

\section{Поглинання і обмеженість}

\begin{definition}
    Нехай~$X$~--- топологічний векторний простір. Будемо говорити, що окіл нуля~$U \in \frak R_0$ \vocab{поглинає} множину~$A \subset X$, якщо існує таке число~$N > 0$, що~$A \subset t U$ для будь-якого~$t \ge N$.
\end{definition}

\begin{definition}
    Множина~$A \subset X$ називається \vocab{обмеженою}, якщо вона поглинається кожним околом нуля.
\end{definition}

\begin{theorem}
    Властивості обмежених підмножин топологічного векторного простору~$X$:
    \begin{enumerate}
        \item Нехай~$A \subset X$~--- обмежена множина. Тоді для будьякого околу~$U \in \frak R_0$ існує таке число~$N > 0$, що~$A \subset t U$ для будь-якого~$t \ge N$.
        \item Об'єднання скінченної кількості обмежених множин обмежене.
        \item Будь-яка скінченна множина є обмеженою.
        \item Будь-який передкомпакт у~$X$ є обмеженим.
    \end{enumerate}
\end{theorem}

\begin{proof}
    \listhack
    \begin{enumerate}
        \item Нехай~$V \in \Omega_0$~--- врівноважений окіл, що міститься в~$U$ за \cref{th:tvs-zero-neighbourhood-properties} (п.~2). Виберемо таке число~$N > 0$, що~$A \subset N V$. Тоді для будь-якого~$t \ge N$ маємо
        \begin{equation*}
            A \subset N V = t(N t\inv V) \subset t V \subset t U.
        \end{equation*}

        \item Нехай~$A_1, A_2, \dots, A_n$~--- обмежені множини, $U$~--- окіл нуля. За попереднім пунктом
        \begin{equation*}
            \forall A_k \quad \exists N_k: \quad \forall t \ge N \quad A_k \subset t U.
        \end{equation*}

        Покладемо~$N = \max_k N_k$, $k = 1, 2, \dots, n$. Тоді~$\forall t \ge N$ усі включення~$A_k \subset t U$ виконуються одночасно, тобто~$\bigcup_{k = 1}^n A_k \subset t U$.

        \item Одноточкова множина є обмеженою, оскільки окіл нуля є поглинаючою множиною. Отже, за попереднім пунктом, будь-яка скінченна множина як скінченне об'\-єд\-на\-ння одноточкових множин є обмеженою.

        \item Нехай~$A$~--- передкомпакт в~$X$, $U$~--- окіл нуля. Виберемо врівноважений окіл~$V \in \Omega_0$, такий що~$V + V \subset U$. За означенням передкомпакта, існує така скінченна множина~$B \subset X$, що~$A \subset B + V$. Відповідно до попереднього пункту, можна знайти такий коефіцієнт~$N > 0$, що~$B \subset N V$. Тоді
        \begin{equation*}
            A \subset B + V \subset NV + V \subset N(V + V) \subset N U. \qedhere
        \end{equation*}
    \end{enumerate}
\end{proof}

\section{Література}

\begin{enumerate}[label={[\arabic*]}]
\item \textbf{Кадец~В.~М.}
Курс функционального анализа /
В.~М.~Кадец~---
Х.: ХНУ им.~В.~Н.~Каразина, 2006. (стр.~502--504).
\end{enumerate}
