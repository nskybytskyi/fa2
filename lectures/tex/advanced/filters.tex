\chapter{Фільтри}

Окрім збіжності напрямленостей, існує ще один вид узагальненої збіжності~--- збіжність фільтрів. Ця ідея базується на альтернативному означенні збіжної послідовності: послідовність $x_n$ називається збіжною до точки $x_0$, якщо для будь-якого околу $U$ цієї точки доповнення до прообразу $f\inv (U)$ є скінченною підмножиною з $\NN$, де $f: \NN \to X$~--- відображення, що задає послідовність. Якщо множину $\NN$ замінити абстрактною множиною $E$, в якому виділено сім'ю підмножин $F$, що має певні загальні властивості, то можна дати розумне означення узагальненої збіжності.

\section{Фільтри}

\begin{definition}
    Сім'я підмножин $\frak F$ множини $X$ називається \vocab{фільтром} на $X$, якщо:
    \begin{enumerate}
        \item Сім'я $\frak F$ непорожня.
        \item $\emptyset \notin \frak F$.
        \item Якщо $A, B \in \frak F$, то $A \cap B \in \frak F$.
        \item Якщо $A \in \frak F$, $A \subset B \subset X$, то $B \in \frak F$.
    \end{enumerate}
\end{definition}

\begin{corollary}
    $X \in \frak F$.
\end{corollary}

\begin{corollary}
    $A_1, A_2, \dots, A_n \in \frak F \implies \bigcap_{i = 1}^n A_i \in \frak F$.
\end{corollary}

\begin{corollary}
    $A_1, A_2, \dots, A_n \in \frak F \implies \bigcap_{i = 1}^n A_i \ne \emptyset$.
\end{corollary}

\begin{example}
    Система $\Omega_x$ усіх околів точки $x$ в топологічному просторі $X$ є фільтром.
\end{example}

\section{Бази фільтрів}

\begin{definition}
    Непорожня сім'я підмножин $\frak D$ множини $X$ називається \vocab{базою фільтра}, якщо:
    \begin{enumerate}
        \item $\emptyset \notin \frak D$;
        \item $\forall A, B \in \frak D$ $\exists C \in \frak D$: $C \subset A \cap B$.
    \end{enumerate}
\end{definition}

\begin{definition}
    Нехай $\frak D$~--- база фільтра. Фільтром, що \vocab{породжений} базою $\frak D$, називається сім'я $\frak F$ усіх множин $A \subset X$, що містять як підмножину хоча б один елемент бази $\frak D$.
\end{definition}

\begin{exercise}
    Довести, що фільтр, породжений базою, дійсно є фільтром.
\end{exercise}

\begin{example}
    Якщо $X$~--- топологічний простір, $x_0 \in X$, $\frak D$~--- сукупність усіх відкритих множин, що містять $x_0$, то фільтр, породжений базою $\frak D$, є фільтром $\frak M_{x_0}$, що складається з усіх околів точки $x_0$.
\end{example}

\begin{definition}
    Нехай $\{x_n\}_{n = 1}^\infty$~--- послідовність елементів множини $X$. Тоді сім'я $\frak D_{\{x_n\}}$ ``хвостів'' послідовності $\{x_n\}_{n = N}^\infty$ є базою фільтра. Фільтр $\frak F_{\{x_n\}}$, породжений базою $\frak D_{\{x_n\}}$, називається фільтром, \vocab{асоційованим} з послідовністю $\{x_n\}_{n = 1}^\infty$.
\end{definition}

\section{Образи фільтрів і баз фільтрів}

\begin{theorem}
    Нехай $X, Y$~--- множини, $f: X \to Y$~--- функція, $\frak D$~--- база фільтра в $X$. Тоді сім'я $f(\frak D)$ усіх множин виду $f(A)$, $A \in \frak D$ є базою фільтра в $Y$.
\end{theorem}

\begin{proof}
    Виконання першої аксіоми бази фільтра є очевидним, адже образ непорожньої множини~--- непорожня множина. Нехай $f(A), f(B)$~--- довільні елементи сім'ї $f(\frak D)$, $A, B \in D$. За другою аксіомою існує таке $C \in \frak D$, що $C \subset A \cap B$. Тоді $f(C) \subset f(A) \cap f(B)$. Отже друга аксіома виконується і для сім'ї $f(\frak D)$.
\end{proof}

\begin{corollary}
    $\frak F$~--- фільтр на $X$, то $f(\frak F)$~--- база фільтра в $Y$.
\end{corollary}

\begin{definition}
    \vocab{Образом фільтра} $\frak F$ при відображенні $f$ називається фільтр $f[\frak F]$, породжений базою $f(\frak F)$, тобто
    \begin{equation*}
        A \in f[\frak F] \iff f\inv (A) \in \frak F.
    \end{equation*}
\end{definition}

\begin{theorem}
    Нехай $\frak C \subset 2^X$~--- непорожня сім'я множин. Для того щоб існував фільтр $\frak F \supset \frak C$ (тобто такий, що усі елементи сім'ї $\frak C$ є елементами фільтра $\frak F$) необхідно і достатньо, щоб $\frak C$ була центрованою.
\end{theorem}

\begin{proof}
    \textbf{Необхідність.} Якщо $\frak F$~--- фільтр і $\frak F \supset \frak C$, то будь-який скінчений набір $A_1, A_2, \dots, A_n$ елементів сім'ї $\frak C$ буде складатися з елементів фільтра $\frak F$. Отже, 
    \begin{equation*}
        \bigcap_{i = 1}^n A_i \ne \emptyset.
    \end{equation*}

    \textbf{Достатність.} Нехай $\frak C$~--- центрована сім'я. Тоді сім'я $\frak D$ усіх множин виду
    \begin{equation*}
        \bigcap_{i = 1}^n A_i, \quad n \in \NN, \quad A_1, A_2, \dots, A_n \in \frak C
    \end{equation*}
    буде базою фільтра. Як фільтр $\frak F$ треба взяти фільтр, породжений базою $\frak D$.
\end{proof}

\section{Фільтри, породжені базою}

\begin{definition}
    Нехай $\frak F$~--- фільтр на $X$. Сім'я множин $\frak D$ називається \vocab{базою фільтра $\frak F$}, якщо $\frak D$ база фільтра і фільтр, породжений базою $\frak D$, збігається з $\frak F$.
\end{definition}

\begin{theorem}
    Для того щоб $\frak D$ була базою фільтра $\frak F$, необхідно і достатньо, щоб виконувалися дві умови:
    \begin{enumerate}
        \item $\frak D \subset \frak F$;
        \item $\forall A \in \frak F$ $\exists B \in \frak D$: $B \subset A$.
    \end{enumerate}
\end{theorem}

\begin{exercise}
    Доведіть цю теорему.
\end{exercise}

\begin{definition}
    Нехай $F$~--- фільтр на $X$ і $A \subset X$. \vocab{Слідом} фільтра $\frak F$ на $A$ називається сім'я підмножин $\frak F_A = \{ A \cap B \mid B \in \frak F\}$.
\end{definition}

\begin{theorem}
    Для того щоб сім'я $\frak F_A$ була фільтром на $A$, необхідно і достатньо, щоб усі перетини $A \cap B$, $B \in \frak F$ були непорожніми.
\end{theorem}

\begin{exercise}
    Доведіть цю теорему.
\end{exercise}

\begin{corollary}
    $\frak F_A$~--- фільтр, якщо $A \in \frak F$.
\end{corollary}

\section{Границі і граничні точки фільтрів}

\begin{definition}
    Нехай на множині $X$ задані фільтри $\frak F_1$ і $\frak F_2$. Говорять, що $\frak F_1$ \vocab{мажорує} $\frak F_2$, якщо $\frak F_2 \subset \frak F_1$, тобто кожний елемент фільтра $\frak F_2$ є водночас і елементом фільтра $\frak F_1$.
\end{definition}

\begin{example}
    Нехай $\{x_n\}_{n \in \NN}$~--- послідовність в $X$, а $\{x_{n_k}\}_{k \in \NN}$~--- її підпослідовність. Тоді фільтр $\frak F_{\{x_{n_k}\}}$ асоційований з підпослідовністю, мажорує фільтр $\frak F_{\{x_n\}}$, асоційований з самою послідовністю. 
    
    Дійсно, нехай $A \in \frak F_{\{x_n\}}$. Тоді існує таке $N \in \NN$, що $\{x_n\}_{n = N}^\infty \subset A$. Але тоді й $\{x_{n_k}\}_{k = N}^\infty \subset A$, тобто $A \in \frak F_{\{x_{n_k}\}}$.
\end{example}

\begin{definition}
    Нехай $X$~--- топологічний простір, $\frak F$~--- фільтр на $X$. Точка $x \in X$ називається \vocab{границею} фільтра $\frak F$ (цей факт позначається як $x = \lim \frak F$), якщо $\frak F$ мажорує фільтр околів точки $x$. Іншими словами, $x = \lim \frak F$, якщо кожний окіл точки $x$ належить фільтру $\frak F$.
\end{definition}

\begin{definition}
    Точка $x \in X$ називається \vocab{граничною точкою} фільтра $\frak F$, якщо кожний окіл точки $x$ перетинається з усіма елементами фільтра $\frak F$. Множина усіх граничних точок фільтра називається $\LIM \frak F$.
\end{definition}

\begin{example}
    Нехай $\{x_n\}_{n \in \NN}$~--- послідовність в топологічному просторі $X$. Тоді $x = \lim \frak F_{\{x_n\}} = \lim_{ n \to \infty} x_n$, а $x \in \LIM \frak F_{\{x_n\}}$ збігається з множиною граничних точок послідовності $\{x_n\}_{n \in \NN}$.
\end{example}

\begin{theorem}
    Нехай $\frak F$~--- фільтр на топологічному просторі $X$, $\frak D$~--- деяка база фільтра $\frak F$. Тоді
    \begin{enumerate}
        \item $x = \lim \frak F \iff \forall U \in \Omega_x$ $\exists A \in \frak D$: $A \subset U$;
        \item $x = \lim \frak F \implies x \in \LIM \frak F$. Якщо до того ж $X$~--- хаусдорфів простір, то у фільтра $\frak F$ немає інших граничних точок. Зокрема, якщо у фільтра в хаусдорфовому просторі є границя, то ця границя є єдиною;
        \item множина $\LIM \frak F$ збігається з перетином замикань усіх елементів фільтра~$\frak F$.
    \end{enumerate}
\end{theorem}

\begin{proof}
    \listhack
    \begin{enumerate}
        \item $x = \lim \frak F$ $\iff$ $\forall U \in \Omega_x$ $U \in \frak F$ $\iff$ $\forall U \in \frak F$ $\exists A \in \frak D$: $A \subset U$.

        \item $x = \lim \frak F, U \in \Omega_x$ $\implies$ $U \in \frak F$ $\implies$ $\forall A \in \frak F$ $A \cap U \ne \emptyset$ $\implies$ $x \in \LIM \frak F$;
        
        $x \in \LIM \frak F$ $\implies$ $\forall U \in \frak F, V \in \Omega_y$ $U \cap V \ne \emptyset$ $\implies$ $x = y$ (оскільки простір хаусдорфів).

        \item $x = \LIM \frak F$ $\iff$ $\forall A \in \frak F, U \in \Omega_x$ $A \cap U \ne \emptyset$ $\iff$ $\forall A \in \frak F$ $x \in \closure A$. \qedhere
    \end{enumerate}
\end{proof}

\begin{theorem}
    Нехай $\frak F_1$, $\frak F_2$~--- фільтри на топологічному просторі $X$ і $\frak F_1 \subset \frak F_2$. Тоді
    \begin{enumerate}
        \item $x = \lim \frak F_1 \implies x = \lim \frak F_2$;
        \item $x \in \LIM \frak F_2 \implies x \in \LIM \frak F_1$;
        \item $x = \lim \frak F_2 \implies x \in \LIM \frak F_1$.
    \end{enumerate}
\end{theorem}

\begin{proof}
    \listhack
    \begin{enumerate}
        \item $\frak F_1$ мажорує фільтр $\frak M_x$ околів точки $x$, $\frak F_1 \subset \frak F_2 \implies \frak M_x \subset \frak F_2$.

        \item Оскільки при збільшенні сім'я множин її перетин зменшується, то
        \begin{equation}
            \LIM \frak F_2 = \bigcap_{A \in \frak F_2} \closure A \subset \bigcap_{A \in \frak F_1} \closure A = \LIM \frak F_1.
        \end{equation}

        \item $x = \lim \frak F_2 \implies x \in \LIM \frak F_2 \implies x \in \LIM \frak F_1$. 
    \end{enumerate}
\end{proof}

\section{Границя функції по фільтру}

\begin{definition}
    Нехай $X$~--- множина, $Y$~--- топологічний простір, $\frak F$~--- фільтр на $X$. Точка $y \in Y$ називається \vocab{границею} функції $f: X \to Y$ по фільтру $\frak F$ (цей факт позначається як $y = \lim_{\frak F} f$, якщо $y = \lim f[\frak F]$. Іншими словами, $y = \lim f [\frak F]$, якщо для довільного околу $U$ точки $y$ існує такий елемент $A \in \frak F$, що $f(A) \subset U$.
\end{definition}

\begin{definition}
    Точка $y \in Y$ називається \vocab{граничною} точкою функції $f: X \to Y$ по фільтру $\frak F$, якщо $y \in \LIM f[\frak F]$, тобто якщо довільний окіл точки $y$ перетинається з образами усіх елементів фільтра $\frak F$.
\end{definition}

\begin{example}
    Нехай $X$~--- топологічний простір, $f: \NN \to X$ і $F$~--- фільтр Фреше на $\NN$. Тоді $\lim_{\frak F} f = \lim_{n \to \infty} f(n)$.
\end{example}

\begin{theorem}
    Нехай $X$ і $Y$~--- топологічні простори, $F$~--- фільтр на $X$, $x = \lim \frak F$ і $f: X \to Y$~--- неперервна функція. Тоді $f(x) = \lim_{\frak F} f$.
\end{theorem}

\begin{proof}
    Нехай $U$~--- довільний окіл точки $f(x)$. Тоді існує окіл $V$ точки $X$, для якого $f(V) \subset U$. Умова $x = \lim \frak F$ означає, що $V \in \frak F$. Інакше кажучи, для довільного околу $U$ точки $f(x)$ ми знайшли шуканий елемент $V \in \frak F$: $f(V) \subset U$.
\end{proof}

\section{Література}

\begin{enumerate}[label={[\arabic*]}]
\item \textbf{Александрян~Р.~А.}
Общая топология /
Р.~А.~Александрян, Э.~А.~Мирзаханян~---
М.: Высшая школа, 1979 (стр.~99--102).
\item \textbf{Кадец~В.~М.}
Курс функционального анализа /
В.~М.~Кадец~---
Х.: ХНУ им.~В.~Н.~Каразина, 2006. (стр.~481--488).
\end{enumerate}
